\documentclass{report}
\pagestyle{plain}
\pagenumbering{arabic}

\usepackage[utf8]{inputenc}
\usepackage[T1]{fontenc}
\usepackage{textcomp}

\usepackage{amsmath}
\usepackage{amsfonts}
\usepackage{amssymb}
\usepackage{amscd,amsmath,amssymb,amstext,amsthm}
\usepackage{braket}
\usepackage{mathpartir}
\usepackage{tikz}
\usepackage[utf8]{inputenc}
\usepackage[T1]{fontenc}
\usepackage{textcomp}
\usepackage{enumitem}
\usepackage{ulem}
\usepackage{booktabs}
\usepackage{setspace}
\usepackage{longtable}
\usepackage{sectsty}
\usepackage{titlesec}
\usepackage{titling}
\usepackage[numbers]{natbib}
\usepackage{graphicx}
\usepackage{caption}
\usepackage{threeparttable}
\usepackage{threeparttablex}
\usepackage{placeins}
\usepackage{subcaption}
\usepackage{textgreek}
\usepackage[version=3]{mhchem}
\usepackage[nottoc]{tocbibind}
\usepackage{xcolor}

\DeclareUnicodeCharacter{2009}{ }

\definecolor{darkblue}{rgb}{0,0,0.7}
\usepackage{hyperref}
\usepackage{url}
%\usepackage{breakurl}
\def\UrlBreaks{\do\/\do-}


\usepackage
[
        a4paper,
        left=30mm,
        right=25mm,
        top=25mm,
        bottom=25mm,    
]
{geometry}

\makeatletter
    \renewcommand{\@makechapterhead}[1]{%
    \vspace*{50 pt}%
    {\setlength{\parindent}{0pt} \raggedright \normalfont
    \bfseries\Huge\sffamily
    \ifnum \value{secnumdepth}>1 
        \if@mainmatter\thechapter.\ \fi%
    \fi
    #1\par\nobreak\vspace{40 pt}}}
    \makeatother

\newcommand{\subtitle}[1]{%
  \posttitle{%
    \par\end{center}
    \begin{center}\large#1\end{center}
    \vskip4em}%
}	

\newcommand*\PrintSkips[1]{%
  \typeout{In #1:}%
  \typeout{\@spaces above: \the\abovecaptionskip}%
  \typeout{\@spaces below: \the\belowcaptionskip}%
}

\newcommand{\tu}{\textmu}
    

\begin{document}
\allsectionsfont{\sffamily}
\onehalfspacing

\setlength\parindent{0pt}

\chapter{Summary}

To do:
\begin{itemize}
	\item Edit description of R package
	\item Add results and discussion part to summary
\end{itemize}

With nearly 10 million deaths in 2020, cancer is one of the leading causes of death worldwide \citep{WHO}\citep{WHO_Cancer}. Its high mortality is predominantly caused by resistance against available cancer therapies [c]. To combat this resistance, new therapies have been developed from which immunotherapeutic approaches have demonstrated promising results [c]. Those immunotherapies make use of engineered or innate immune cells, especially T cells, to eliminate malignancies [c]. Nevertheless, solid cancer types often remain resistant due to their immunosuppressive tumour microenvironment [c]. To overcome this hurdle, it is essential to understand the crucial factors promoting successful T cell killing. 

To deepen our understanding of T cell killing, there exist a wide range of \textit{in vitro} and \textit{in vivo} methods. A very intuitive method is to \textit{watch} the T cells while interacting with tumour cells by using live cell imaging methods combined with a range of labelling techniques. For this, the OVA (ovalbumin)-tumour model comes handy as it provides tumour cell lines expressing the OVA antigen and transgenic mice expressing OVA-targeting T cells. 

To investigate the interaction between T cells and tumour cells, the quantitative analysis of large data sets is essential. \textcolor{cyan}{ For this, we built the R package \textit{cellcontacts} which processes the output from cell segmentations tool like StarDist and the cell tracking tool TrackMate [cc]. Our R package offers a range of functions to compute cell-cell  contacts, characteristics of cell-cell contacts and it allows to identify cell death and to connect the results to immunological staining results.}

To put it in a nutshell, this thesis' aim was to widen the explanatory power of available live cell imaging techniques by enabling large-scale analyses of cell interaction and correlation to immunological staining methods. We show that [...]

%%===================================================
\chapter{Introduction}

%\begin{itemize}
%	\item Why T cells are crucial for tumour elimination
%	\item Crucial factors for T cell killing and the relevance of cell dynamics
%	\item Tools to investigate cell dynamics
%	\item Software to analyze cell tracks
%\end{itemize}

In this section, we will introduce the essential role of T cells for tumour cell elimination and afterwards, we will focus on the crucial factors for efficient T cell killing. We will explain the necessity to understand cell dynamics for unravelling T cell killing mechanisms. Eventually, we will address available tools to analyse cell movement.

\section{T cells are crucial for tumour elimination}

To do:
\begin{itemize}
	\item Mention anti-tumour activity of NK cells
	\item Compare NK cells vs. T cells
\end{itemize}

Cancer is a disease in which body cells grow uncontrollably because of inherited or acquired mutations [c]. Those cells form tissue mass, called malignant tumours, that can spread over the body, impede normal body cells and cause disease [c].
Naturally, the immune system is capable to track down and eliminate such abnormal cells by inducing their cell death. For this, T cells, a type of immune cell, use their surface antigen receptors to find irregularities within the surface antigens of abnormal cells. If they find suspicious cells, T cells induce their apoptosis, a well-regulated cell death.
In accordance, malfunction of the immune system is strongly correlated to a higher incidence of cancer [ccc]\citep{RN295}. In particular, a decline in T cell mediated immune response can lead to a reduced response against malignant tumour cells and cause disease \citep{RN295}.
In conclusion, if our immune mechanisms to eliminate abnormal cells are suppressed or fail, cancer cells can form malignant tumours. Knowing these mechanisms and the consequences of their failure, it is clear that T cells play a key role for the elimination of tumour cells.

\section{CAR T cell therapy exploits the killing activity T cells}

To-do:
\begin{itemize}
	\item Add some statistics and citations on antibody and CAR T cell therapy
\end{itemize}

Because T cells are capable of eliminating tumour cells, it is apparent that one would like to enhance their killing activity to prevent and treat cancer. To achieve this goal, we first need to understand why natural T cells are not always capable of full tumour elimination: Researchers found that tumour cells can escape immuno-surveillance through inhibiting T cell activity, down-regulating their surface antigens, and inducing T cell exhaustion [ccc]. 

Understanding these immune escape mechanisms, allowed two major breakthroughs in cancer therapy: 
Check-point inhibitors were discovered which can reverse an inhibition on T cell activity. The release of this brake allows T cells to become active and track down and kill cancer cells [ccc]. This finding was awarded with a Nobel Prize in 2018 and the consequent therapeutic approaches, called antibody therapies, became a standard in the clinics [c]. 

Even with a released brake, T cells are not able to track down tumour cells that hide by down-regulating their surface-antigens. To tackle this obstacle, researchers have genetically engineered T cells that  carry modified antigen receptors such that they can identify and kill hiding tumour cells [c]. This approach, called chimeric antigen receptor (CAR) T cell therapy, was successful in treating leukaemia cases that were incurable before [cc]. Hence, it was a second major breakthrough in cancer therapy [cc]. 

Following the great success of CAR T cell therapy to treat haematological malignancies, efforts have sparked to translate this approach to solid tumour entities [ccc]. But in contrary to leukaemia, solid tumours have a often times stiff tumour micro-environment which consists of an extracellular matrix filled different non-cancerous cell types [ccc]. This makes it difficult for T cells and CAR T cells to even enter the tumour tissue before they have a chance to identify and kill the tumour cells [ccc]. Once T cells got through the tumour micro-environment and arrived at the tumour cells, they are usually exhausted and their killing activity is drastically reduced [cc]. 

To increase the success of CAR T cell therapy in solid malignancies, one could either tackle the tumour micro-environment or further modify CAR T cells such that they become more effective even in an exhaustive environment. There is much research approaching both ideas and adequate methods are required to evaluate the success of those modifications [cc]. In the following section, we will put our focus on the methods to evaluate the success of tumour treatments. Furthermore, we will address why the investigation of cell dynamics is pivotal to evaluate the success of T cell killing.


\section{Why are cell dynamics relevant?}

\begin{itemize}
	\item Describe tools to evaluate CAR T cell efficiency
	\item Name the advantages and also shortcomings of static methods (FACS, sequencing, immuno-stainings)
	\item Name the benefit of cell dynamics
	\item In particular:
		\begin{itemize}
	\item Explain why understanding cell-cell interactions are essential for tumour killing
	\item What is known about cell-cell interaction and what we would like to know about cell-cell interaction
\end{itemize}

\end{itemize}

To observe the success of T cell killing, the most straightforward approach would be to measure the amount of surviving tumour cells before and after therapy. But this is neither sufficient nor does this method reveal the full picture: It does not give an insight into [name more reasons]\\


Hence, the interaction between T cells, tumour cells and their tumour micro-environment should be observed from different angles which we can roughly classify into static and dynamic methods: Static methods give a detailed insight into the cellular processes at a specific point, including the gene and protein expression profile of cells on a single cell-level, the distribution of cells in tissue and the cells' state in the cell cycle [ccc]. 
%Examples for static methods include flow cytometry, DNA- and RNA-sequencing techniques and immunological staining of tissue and cell samples. 
On the other hand, dynamical methods are able to capture immune and tumour cells on a temporal scale. They widen our understanding of cell-cell-interaction, cell growth and killing activity [ccc]. For example, \textit{in vitro} live-cell imaging can visualize how tumour cells move during tumour growth and invasion and with additional labelling of $\text{Ca}^{2+}$-signalling, we can investigate \textit{when and how} the T cells try to kill tumour cells. [Mention additive cytotoxiciy and maybe also other papers] \\


Given the different aspects that static and dynamic methods elucidate, it is beneficial to join forces to understand and modify existing tumour therapies. For this, we need comprehensive analysis work flows that combine results from static and dynamic experiments.

In the following section, we will give an overview of existing tools to investigate cell dynamics in live-cell imaging films. In the end, we will give an insight into available tools to combine static with dynamic methods to understand T cell killing mechanisms.

\section{Tools to investigate cell dynamics}

Cell dynamics are equally important as static cell information. In this project, we will first put our focus on methods to quantitatively analyse cell dynamics and cell-cell interaction. Following, we will address methods to connect results on cell dynamics to the findings from static experiments.

Live-cell imaging techniques usually produce a file of around 35 GB file size with around 3000 frames capturing a region of interest of around $\color{cyan}{x} \times \color{cyan}{x}$ \textmu m with around \textcolor{cyan}{x} cells for \textcolor{cyan}{x} hours. After acquiring such a data set, it is evident that it is not feasible to analyse the whole data set manually. Therefore, tools which can automatically detect cells and track their movement are indispensable. In this section, we will introduce available tools for automated cell segmentation and tracking.

\subsection{Cell segmentation}

\begin{itemize}
	\item Stardist
	\item CellPose
\end{itemize}

Before analysing cell movement, cells need to be automatically identified and their boundary accurately delineated, called segmented. For this purpose, several cell segmentation tools have been published [ccc]. Within these tools, StarDist has proved to be reliable for an automatic detection of cell nuclei in a regular roundish shape. The core of StarDist is a convolutional network which has been trained to predict the occurrence of a polygon representing the cell nucleus shape for every pixel. This approach facilitates the correct segmentation of crowded cells which is a typical phenomenon in live cell imaging although it does not allow the segmentation of ring- or in general odd-shaped cell nuclei \citep{RN285}. 
%Another cell-segmentation tool is CellPose

\subsection{Cell tracking}

\begin{itemize}
	\item Overview and classification
	\item LAP tracker
	\item Kalman tracker
\end{itemize}

Regarding the tracking of cell movement, there exist a broad range of algorithms [cc]. A comprehensive analysis of available cell-tracking algorithms was given by Ulman \textit{et al} in which 21 cell-tracking algorithms were classified regarding their tracking strategy and evaluated based on their performance on up to 13 data sets.
In terms of the tracking strategy, the algorithms either used cell contour evolution techniques or cell detection methods. The former requires a spatio-temporal overlap between the corresponding cell regions, thus performing segmentation and tracking in one task, whereas the latter firstly segments the cells in all frames and afterwards, builds a probabilistic framework to propose temporal associations between the cells. While some tracking algorithms achieved good results, none of them achieved fully correct solutions for all data sets. Hence, a major implication of this report is that the right choice of tracking algorithm heavily depends on the applied imaging modality and the respective data set \citep{RN292}. 

[More information on T cell tracking, segmentation + detection and not cell contour evolution]

To facilitate the usage of cell-tracking algorithms, the extensible, open-source Fiji plugin \textit{TrackMate} was published by Tinevez \textit{et al} \citep{RN293}. Via a graphical user interface, it allows the user to choose a segmentation algorithm and a tracking algorithm for their specific data set in any combination. The available segmentation algorithms included ilastik, Weka, cellpose, MorphoLibJ and StarDist. Moreover, TrackMate allows the import of segmentation results as mask or label images. For the tracking algorithm, TrackMate offers the LAP framework by Jaqaman \textit{et al} for Brownian motion [c] and the Kalman filter for linear motion [c]. Overall, TrackMate offers an open-source framework to combine any segmentation and tracking algorithm for a customized cell-tracking work flow.


\subsection{Cell-cell contacts}

\begin{itemize}
	\item Available tools
	\item Our needs -> quantitative information from microscopy images of cultured cells
\end{itemize}

To gain biological insights from time-lapse microscopy recordings of cell behaviour, ...


How do the dynamics of T cell behaviour correlate to tumour killing efficiency? This is the question central to this thesis. 


%===================================================
\bibliographystyle{unsrtnat}
\renewcommand{\bibname}{References}
\bibliography{literatur}


\end{document}