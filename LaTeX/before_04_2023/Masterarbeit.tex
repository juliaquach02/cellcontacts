\documentclass{report}
\pagestyle{plain}
\pagenumbering{arabic}

\usepackage[utf8]{inputenc}
\usepackage[T1]{fontenc}
\usepackage{textcomp}

\usepackage{amsmath}
\usepackage{amsfonts}
\usepackage{amssymb}
\usepackage{amscd,amsmath,amssymb,amstext,amsthm}
\usepackage{braket}
\usepackage{mathpartir}
\usepackage{tikz}
\usepackage[utf8]{inputenc}
\usepackage[T1]{fontenc}
\usepackage{textcomp}
\usepackage{enumitem}
\usepackage{ulem}
\usepackage{booktabs}
\usepackage{setspace}
\usepackage{longtable}
\usepackage{sectsty}
\usepackage{titlesec}
\usepackage{titling}
\usepackage[numbers]{natbib}
\usepackage{graphicx}
\usepackage{caption}
\usepackage{threeparttable}
\usepackage{threeparttablex}
\usepackage{placeins}
\usepackage{subcaption}
\usepackage{textgreek}
\usepackage[version=3]{mhchem}
\usepackage[nottoc]{tocbibind}

\DeclareUnicodeCharacter{2009}{ }

\definecolor{darkblue}{rgb}{0,0,0.7}
\usepackage{hyperref}
\usepackage{url}
\usepackage{breakurl}
\def\UrlBreaks{\do\/\do-}


\usepackage
[
        a4paper,
        left=30mm,
        right=25mm,
        top=25mm,
        bottom=25mm,    
]
{geometry}

\makeatletter
    \renewcommand{\@makechapterhead}[1]{%
    \vspace*{50 pt}%
    {\setlength{\parindent}{0pt} \raggedright \normalfont
    \bfseries\Huge\sffamily
    \ifnum \value{secnumdepth}>1 
        \if@mainmatter\thechapter.\ \fi%
    \fi
    #1\par\nobreak\vspace{40 pt}}}
    \makeatother

\newcommand{\subtitle}[1]{%
  \posttitle{%
    \par\end{center}
    \begin{center}\large#1\end{center}
    \vskip4em}%
}	

\newcommand*\PrintSkips[1]{%
  \typeout{In #1:}%
  \typeout{\@spaces above: \the\abovecaptionskip}%
  \typeout{\@spaces below: \the\belowcaptionskip}%
}

\newcommand{\tu}{\textmu}
    
\begin{document}
\allsectionsfont{\sffamily}
\onehalfspacing


\title{\sffamily WORKING TITLE\\ Developing a pipeline for high-throughput analysis of dynamic and static T cell/tumor cell data \vskip4em}
\subtitle{\sffamily
 \large \textbf{Masterarbeit\\ and der Medizinischen Fakultät \\der Eberhard Karls Universität Tübingen}}
\author{\large \sffamily vorgelegt von \vspace{2ex}\\ \sffamily \textbf{Quach, Julia}}
\date{\large \sffamily\textbf{Tübingen, DATUM}}
\maketitle

%\chapter*{Statement}
%
%I hereby declare:
%\begin{itemize}
%\item that this thesis is my own work and that I have not made use of any other sources or aids than those referenced.
%\item that all statements appropriated from other works in letter or in substance have been referenced as such. 
%\item that the thesis submitted has not been the subject of any other examination in essential parts or in full
%\item that the thesis submitted has not been published in essential parts or in full.
%\end{itemize}
%
%
%\noindent I affirm that I have written the dissertation myself and have not used any sources and
%aids other than those indicated.
%The references are presented accorded to standard rules for publication and
%standard citation guidelines.\\\\
%
%
%\noindent Tübingen, Date, Signature




\tableofcontents
%===================================================

\setlength\parindent{0pt}

\chapter{Summary}

It is widely known that cancer is one of the leading causes of death in Western society [citation]. Its high mortality is predominantly caused by tumor resistance against all available therapies [citation]. To combat this resistance, new therapies have been developed from which the most recent and promising therapeutic approach are immunotherapies which have shown great success in the past [citation]. Immunotherapies make use of engineered or innate immune cells, especially T cells, to eliminate malignant tumor cells. Still, many cancer types remain resistant. To overcome those therapy resistances, there follows a high need to understand the crucial factors behind efficient T cell killing. \\

To elucidate the mechanisms of efficient T cell killing, [Introduction on OVA tumor models, live cell imaging, staining, fix-while-filming ]\\

To achieve meaningful results, large data sets and their statistical analysis are pivotal. To enable a large-scale analysis of tumor and T cell interaction, this thesis aims at developing an R package to automatically evaluate cell dynamics in live cell imaging movies. This would enable a quantitative analysis of cell-cell contacts on big sample sets. Besides, the R packages proposes to correlate dynamical cell data to immunological staining results. To put it in a nutshell, this thesis' goal is to heighten the explanatory power of available live cell imaging and immunological staining techniques by enabling large-scale analyses and correlation of dynamical to immunological staining methods.\\

To achieve an efficient analysis of live cell imaging data, [... More details on the R package and workflow]. \\

We show that [...]

\chapter{Introduction}

\begin{itemize}
	\item T cell activity against tumor cells
	\item Crucial factors for T cell killing and the relevance of cell dynamics
	\item Tools to investigate cell dynamics
	\item Software to analyze cell tracks
\end{itemize}

\section{T cells against tumor cells}

\begin{itemize}
	\item Hallmarks of Cancer
	\item Escaping immune system
	\item T cell killing
	\item Tumor cell elimination by the body always by T cells (I think)
\end{itemize}

\section{The relevance of cell dynamics}

\begin{itemize}
	\item Advantages of static methods (FACS, sequencing, immuno-stainings)
	\item Shortcomings of static methods
	\item Benefit of cell dynamics
\end{itemize}

\section{Available tools to investigate cell dynamics}

Requires recognizing the cell and then characterizing its movement.

\subsection{Cell segmentation}

\begin{itemize}
	\item Stardist
	\item CellPose
\end{itemize}

\subsection{Cell tracking}

\begin{itemize}
	\item LAP tracker
	\item Kalman tracker
\end{itemize}

\section{ Scope }

\chapter{Methods}

\begin{itemize}
	\item Imaging
	\item Segmentation
	\item Tracking
	\item Export and loading into RStudio
	\item Cell-cell contact computation
	\item Quality control (CelltrackR)
\end{itemize}

\section{Extracting features}

For a given set of cell tracks, we would like to observe meaningful features of the cell movement and activity. To acquire the cell motility data, tumor cells (B16F10-H2BmCherry-OVA) and T cells (OT1 GFP) were seeded into a 3D gel and imaged over time for several hours at a rate of 90 seconds per image. Afterwards, the images were segmented and cell tracks were obtained using an image analysis tool, e.g., the TrackMate Plugin for ImageJ or Imaris 3D.\\

The cell tracks have the following structure: Each .csv-file is a table with the columns "ID", "time point", "x-" , "y-" and "z"-coordinate. We have separate .csv-files for the tracks from tumor cells and the tracks from T cells.

\begin{itemize}
	\item  $X :=$ Tracks from tumor cells (tumorTracks.csv)
	\item $Y :=$ Tracks from T cells (tcellTracks.csv)
\end{itemize}

We would like to extract the following features from our data:
\begin{enumerate}
	\item Identify cell-cell contacts between T cells and tumor cells
	\item Number of T cell contacts per tumor cells
	\item Number of contacts between one T cells and one tumor cells
	\item Duration of a T cell contact
	\item Correlation of T cell dynamics to tumor cell outcome (i.e., Fix-While-Filming method).
\end{enumerate}

\subsection{Identify cell-cell contacts between tumor and melanoma cells.}
\label{contacts(X,Y,radDist,minTime)}

We implement a function \textit{contacts}(X,Y, radDist, minTime) where $X:=$ tumorTracks.csv and $Y:=$ tcellTracks.csv are tracks objects, \textit{radDist} is a distance threshold and \textit{minTime} a minimum of consecutive timepoints.\
The function shall return an array (in Python it should be a list) of cell pairs that have a distance below radDist for more than minThresh timepoints with the start and end time of this "contact".

\begin{enumerate}
	\item Calculate the Euclidean distance for every cell pair $(x,y)$ where $x \in X$ and $y \in Y$ for every timepoint $t$.
	\item Set threshold $\text{radDist}$ for the radial distance.
	\item Filter for cell pairs with a distance below radDist.
	\item Set threshold $\text{minTime}$ for the number of consecutive timepoints.
	\item Filter for cell pairs with a distance below radDist for more than minTime timepoints. 
	\item Return a list with cell pairs and their start and end time for which their distance is below the given threshold.
\end{enumerate}

\underline{Note:} To code this, use the Matlab (and the R) implementation as pseudo code.

\subsection{Number of T cell contacts per tumor cell}
\label{tContactsPerCell(X,Y, radDist, minTime)}

We implement a function \textit{numContacts}(X,Y, radDist, minTime).

\begin{enumerate}
	\item Apply \textit{contacts}(X, Y, radDist, minTime)
	\item Sort by elements from $X$ (tracks of tumor cells).
	\item For every tumor cell $x_i\in X$ that was involved in a contact, count the number of contacts and return it with the respective start and end time of contact (duration).
\end{enumerate}

\subsection{Number and duration of contacts for one T cell and one tumor cell}

We implement a function \textit{numContactsDetailed}(X,Y, radDist, minTime).

\begin{enumerate}
	\item Apply \textit{numContacts}(X, Y, radDist, minTime)
	\item Return two lists:
		\begin{enumerate}
			\item A list of the tumor cells and for each tumor cell, the number and respective duration of contacts with all its contact T cells.
			\item  A list of the T cells and for each T cell, the number and respective duration of contacts with all its contact tumor cells.
		\end{enumerate}
\end{enumerate}


\subsection{Correlation of  T cell dynamics to tumor cell outcome}

Analysis of the tracks object:

\begin{enumerate}
	\item Apply \textit{contacts}(X, Y, radDist, minTime) = result.
	\item For every tumor cell $x\in$ result, save its ID, its last timepoint and its xyz-coordinate.
	\item Exclude the cells which did not last until the last timepoint.
	\item Return a list \textit{Candidates} with the cell IDs, its last timepoint and its xyz-coordinate.
\end{enumerate}

Analysis of the fixed sample:

\begin{enumerate}
	\item For every parameter (p21, etc.), set a ROI around the positive cells.
	\item Convert each ROI to a set $A_i \subseteq \mathbb{R}^2$ and label the ROIs uniquely.
	\item For each ROI $A_i$, check whether a cell $x\in\textit{Candidates}$ is also in our ROI $x\in A_i$.
	\item Allow only one cell per ROI.
	\item For each parameter, return a list of cell IDs that matched a ROI.
\end{enumerate}

\section{Data for network training}

Our feature vector is: For each tumor cells in $x_0 \in X$.

\begin{equation*}
	\begin{pmatrix}
		\text{T cell contacts to } x_0  \text{ in total (integer)}	\\
		\text{Array of contacts of } x_0 \text{ to each T cells (boolean array)}\\
		\text{Array of contact duration to each T cell (float array)} \\
	\end{pmatrix}
\end{equation*}

The length of the latter two arrays is the total number of T cells.

Our outcome vector is: For each tumor cells in $x_0 \in X$.
\begin{equation*}
 \left(
	\begin{array}{c}
		\text{dead (yes/no)} \\
		\text{p16 (yes/no)}		\\
		\text{p21 (yes/no)}		\\
		\text{End nucleus size}		\\
		\text{Ratio start and end nucleus size}		\\
		\text{Parameter } [\dots]		\\
	\end{array}
\right) 
\end{equation*}

where the parameter \textit{dead},  \textit{p16} and \textit{p21} is a boolean, and the other parameters can be of the float type (e.g., representing fluorescence intensity of a marker).


\section{Improvements}

\subsection{Improvement of tracking}
Instead of calculating the distance between xyz-coordinates where one xyz-coordinate represents one cell, calculate the distance between ROIs.

\subsection{Extracting senescence-relevant information (Oli)}

If we can extend our data:
\begin{itemize}
	\item Tracks from tumor cells (tumorTracks.csv)
	\item Tracks from T cells (tcellTracks.csv)
	\item \textcolor{blue}{Tracking of nucleus size: Time points of cell tracks for which an enlarged nucleus is detected}
\end{itemize}

Then, we would like to extract the following additional features from our data:

\begin{enumerate}
	\item For each tumor cell, Number of T cell contacts before and after change of nucleus size. 
	\item For each tumor cell, duration between last CTL contact and nucleus change.
	\item Behavior of surrounding cells when senescent cell is present.
\end{enumerate}

\paragraph{Number of T cell contacts before and after change of nucleus size}

Under the assumption that cells do not switch from a senescent state back to a normal state: For each tracks object $t_j$, $j\in J = {All tracks}$:

\begin{enumerate}
 \item Split the tracks object $t_j$ at the first time point when an enlarged nucleus is detected into two tracks objects $t_{j1}$ and $t_{j2}$
 \item Calculate the number of T cell contacts with \textit{contacts}(X,Y, radDist, minTime) for $t_{j1}$ and $t_{j2}$.
 \item Compare the number of T cell contacts for $t_{j1}$ and $t_{j2}$.
\end{enumerate}

\paragraph{Duration between last CTL contact and nucleus change}

In our data set, we should have the time point of the nucleus change. Now, we have to find the time point of the last CTL contact before nucleus change.\\

For each tracks object $t_j$, $j\in J = {All tracks}$:

\begin{enumerate}
	\item Split the tracks object $t_j$ at the first time point when an enlarged nucleus is detected into two tracks objects $t_{j1}$ and $t_{j2}$
	\item For $t_{j1}$ (which is the track before the nucleus size changes), use the function to calculate the duration of T cell contact per tumor cell (see \ref{Duration of T cell contact per tumor cell}).
	\item For the last contact in $t_{j1}$, get the last time point.
	\item Return the difference between the time point of the nucleus change and the last time point of the last contact.
\end{enumerate}


\paragraph{Behavior of surrounding cells when senescent cell is present}

Apply the following function for conditions with and without CTL:

\begin{enumerate}
	\item For a tumor cell $x \in X$, consider its surrounding cells within a radius $r$.
	\item Count the number of surrounding tumor cells that turn senescent.
\end{enumerate}

\subsection{Additional data for network training}

Our additional features for each tumor cells in $x_0 \in X$.

\begin{equation*}
 \left(
	\begin{array}{c}
		\text{Number of T cell contacts before change of nucleus}\\
		\text{Number of T cell contacts after change of nucleus}\\
		\text{Ratio of T cell contacts before and after change of nucleus}\\
		\text{Duration between last T cell contact and nucleus change}\\
		\text{Cumulative number of surrounding cells with enlarged nucleus}\\
	\end{array}
\right) 
\end{equation*}


\chapter{Results}

An R package was developed with the following dependencies and functions:\\

Dependencies:
\begin{itemize}
 \item celltrackR,
    dplyr,
    ggplot2,
    keep,
    pracma,
    stringr
\end{itemize}

Functions:
\begin{itemize}
	\item cellContactsByTracks
	\item cellContactsByROIs
\end{itemize}

Überarbeiten für ROI und Track output:
\begin{itemize}
	\item contactByPairs
	\item displayCellDuringContact
	\item displayContacts
	\item matchTrackNamesWithIDs
	\item numContactsPerCell
	\item numContactsperPair
	\item plotTracksWithROIs
	\item ROIsToTracks
	\item speedDuringContact
\end{itemize}


%%===================================================
%\chapter{Discussion}
%
%
%%===================================================
%\chapter{Conclusion}
%
%
%%===================================================
%\chapter*{Acknowledgment}
%\addcontentsline{toc}{chapter}{Acknowledgment}



%===================================================
%
%\listoffigures
%\listoftables
%%===================================================
%\bibliographystyle{unsrtnat}
%\renewcommand{\bibname}{References}
%\bibliography{literatur}
%
%
%
%%===================================================
%\chapter*{Supplement}
%\addcontentsline{toc}{chapter}{Supplementary}
%
%\newcommand{\beginsupplement}{
%        \setcounter{table}{0}
%        \renewcommand{\thetable}{S\arabic{table}}%
%        \setcounter{figure}{0}
%        \renewcommand{\thefigure}{S\arabic{figure}}%
%     }
% 
%\beginsupplement


\end{document}