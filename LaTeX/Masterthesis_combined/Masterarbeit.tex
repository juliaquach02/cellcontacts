\documentclass{report}
\pagestyle{plain}
\pagenumbering{arabic}

\usepackage[utf8]{inputenc}
\usepackage[T1]{fontenc}
\usepackage{textcomp}

\usepackage{amsmath}
\usepackage{amsfonts}
\usepackage{amssymb}
\usepackage{amscd,amsmath,amssymb,amstext,amsthm}
\usepackage{braket}
\usepackage{mathpartir}
\usepackage{tikz}
\usepackage[utf8]{inputenc}
\usepackage[T1]{fontenc}
\usepackage{textcomp}
\usepackage{enumitem}
\usepackage{ulem}
\usepackage{booktabs}
\usepackage{setspace}
\usepackage{longtable}
\usepackage{sectsty}
\usepackage{titlesec}
\usepackage{titling}
\usepackage[numbers]{natbib}
\usepackage{graphicx}
\usepackage{caption}
\usepackage{threeparttable}
\usepackage{threeparttablex}
\usepackage{placeins}
\usepackage{subcaption}
\usepackage{textgreek}
\usepackage[version=3]{mhchem}
\usepackage[nottoc]{tocbibind}

\DeclareUnicodeCharacter{2009}{ }

\definecolor{darkblue}{rgb}{0,0,0.7}
\usepackage{hyperref}
\usepackage{url}
\usepackage{breakurl}
\def\UrlBreaks{\do\/\do-}


\usepackage
[
        a4paper,
        left=30mm,
        right=25mm,
        top=25mm,
        bottom=25mm,    
]
{geometry}

\makeatletter
    \renewcommand{\@makechapterhead}[1]{%
    \vspace*{50 pt}%
    {\setlength{\parindent}{0pt} \raggedright \normalfont
    \bfseries\Huge\sffamily
    \ifnum \value{secnumdepth}>1 
        \if@mainmatter\thechapter.\ \fi%
    \fi
    #1\par\nobreak\vspace{40 pt}}}
    \makeatother

\newcommand{\subtitle}[1]{%
  \posttitle{%
    \par\end{center}
    \begin{center}\large#1\end{center}
    \vskip4em}%
}	

\newcommand*\PrintSkips[1]{%
  \typeout{In #1:}%
  \typeout{\@spaces above: \the\abovecaptionskip}%
  \typeout{\@spaces below: \the\belowcaptionskip}%
}

\newcommand{\tu}{\textmu}
    
\begin{document}
\allsectionsfont{\sffamily}
\onehalfspacing


\title{\sffamily Developing the R package \textit{cellcontacts} for the high-throughput analysis of cell-cell interaction between T cells and tumour cells \vskip4em}
\subtitle{\sffamily
 \large \textbf{Masterarbeit\\ and der Medizinischen Fakultät \\der Eberhard Karls Universität Tübingen}}
\author{\large \sffamily vorgelegt von \vspace{2ex}\\ \sffamily \textbf{Quach, Julia}}
\date{\large \sffamily\textbf{Tübingen, DATUM}}
\maketitle

%\chapter*{Statement}
%
%I hereby declare:
%\begin{itemize}
%\item that this thesis is my own work and that I have not made use of any other sources or aids than those referenced.
%\item that all statements appropriated from other works in letter or in substance have been referenced as such. 
%\item that the thesis submitted has not been the subject of any other examination in essential parts or in full
%\item that the thesis submitted has not been published in essential parts or in full.
%\end{itemize}
%
%
%\noindent I affirm that I have written the dissertation myself and have not used any sources and
%aids other than those indicated.
%The references are presented accorded to standard rules for publication and
%standard citation guidelines.\\\\
%
%
%\noindent Tübingen, Date, Signature


\tableofcontents
%===================================================

\setlength\parindent{0pt}

\chapter{Summary}

It is widely known that cancer is still one of the leading causes of death in Western society [citation]. Its high mortality is predominantly caused by tumor resistance agains available therapies [citation]. To combat this resistance, new therapies have been developed from which the most recent and promising therapeutic approach are immunotherapies which have shown great success in the past [citation]. Immunotherapies make use of engineered or innate immune cells, especially T cells, to eliminate malignant tumor cells. Still, many cancer types remain resistant. To overcome this, there follows a high need to understand the crucial factors behind therapy resistance and efficient T cell killing. \\

To elucidate the mechanisms of efficient T cell killing, [Introduction on OVA tumor models, live cell imaging, staining, fix-while-filming ]\\

To achieve meaningful results, large data sets and their statistical analysis are pivotal. To enable a large-scale analysis of tumor and T cell interaction, this thesis aims at developing an R package to automatically evaluate cell dynamics in live cell imaging movies. This would enable a quantitative analysis of cell-cell contacts on big sample sets. Besides, the R packages proposes to correlate dynamical cell data to immunological staining results. To put it in a nutshell, this thesis' goal is to heighten the explanatory power of available live cell imaging and immunological staining techniques by enabling large-scale analyses and correlation of dynamical to immunological staining methods.\\

To achieve an efficient analysis of live cell imaging data, [... More details on the R package and workflow]. \\

We show that [...]

%%===================================================
\chapter{Introduction}
\begin{itemize}
	\item Why T cells are crucial for tumour elimination
	\item Crucial factors for T cell killing and the relevance of cell dynamics
	\item Tools to investigate cell dynamics
	\item Software to analyze cell tracks
\end{itemize}

\section{T cells are crucial for tumour elimination}

\begin{itemize}
	\item Hallmarks of Cancer
	\item Escaping immune system
	\item T cell killing
	\item Tumor cell elimination by the body always by T cells (I think)
\end{itemize}

\section{CAR T cells as arising tumour therapy}

\begin{itemize}
	\item Introduction to CAR T cells
	\item Potential of CAR T cells and shortcomings so far
	\item Approaches to improve CAR T cells (genetic, antigen receptor)
	\item Tools to evaluate CAR T cell efficiency (and why dynamic information is important)
\end{itemize}

\section{The relevance of cell dynamics}

\begin{itemize}
	\item Advantages of static methods (FACS, sequencing, immuno-stainings)
	\item Shortcomings of static methods
	\item Benefit of cell dynamics
	\item Results so far about cell dynamics
	\item Aim of this thesis
\end{itemize}

\paragraph{Cell-cell interaction is essential for tumour killing}

\begin{itemize}
	\item What is known about cell-cell interaction
	\item Why it is important to investigate cell-cell interaction
	\item What we would like to know about cell-cell interaction
\end{itemize}

This is the question central to this thesis. How do the dynamics of T cell behaviour correlate to tumour killing efficiency?
More specifically, this thesis builds a tool for a large-scale analysis of cell-cell interaction between T cells and tumour cells. Furthermore, we combine our results on dynamical data to findings on tumour killing efficiency from immunological methods.

\section{Tools to investigate cell dynamics}

Requires recognizing the cell and then characterizing its movement.
Live Cell Imaging

\subsection{Cell segmentation}

\begin{itemize}
	\item Stardist
	\item CellPose
\end{itemize}

\subsection{Cell tracking}

\begin{itemize}
	\item LAP tracker
	\item Kalman tracker
\end{itemize}

\subsection{Cell-cell contacts}

\begin{itemize}
	\item Available tools
	\item Our needs
\end{itemize}

%%===================================================
\chapter{Methods}

\begin{itemize}
	\item Overview
	\item Imaging details
	\item Segmentation
	\item Tracking
	\item Export to RStudio
	\item Cell-cell contact computation
	\item Extracting more features
	\item Quality control (CelltrackR)
	\item Data analysis
\end{itemize}

\section{Work flow}

\section{Extracting features}

For a given set of cell tracks, we would like to observe meaningful features of the cell movement and activity. To acquire the cell motility data, tumour cells (B16F10-H2BmCherry-OVA) and T cells (OT1 GFP) were seeded into a 3D collagen gel and imaged over time for several hours at a rate of 90 seconds per image. Afterwards, the images were segmented and cell tracks were obtained using an image analysis tool, e.g., the TrackMate Plugin for ImageJ or Imaris 3D.\\

The cell tracks have the following structure: Each .csv-file is a table with the columns "ID", "time point", "x-" , "y-" and "z"-coordinate. We have separate .csv-files for the tracks from tumour cells and the tracks from T cells.

\section{Developing the R package \textit{cellcontacts}}

\subsection{ Computing cell-cell contacts }

\subsection{ Identifying dying tumour cells}

\section{Tools for quality control}

%%===================================================
\chapter{Results}

\section{Segmentation and tracking of cells}

\section{Computing cell-cell contacts}

\section{Benchmarking computation time}

\section{Investigation of killing efficiency}

\begin{itemize}
	\item Identifying dead cells
\end{itemize}

\section{Investigating the induction of cell senescence}

\begin{itemize}
	\item Visualizing cell size with regard to cell-cell contacts
\end{itemize}


%%===================================================
\chapter{Discussion}

\section{Biological interpretation}

\section{Working title: Comparison to available tools}

%%===================================================
%\chapter{Conclusion}
%
%
%%===================================================
%\chapter*{Acknowledgment}
%\addcontentsline{toc}{chapter}{Acknowledgment}



%===================================================
%
%\listoffigures
%\listoftables
%%===================================================
%\bibliographystyle{unsrtnat}
%\renewcommand{\bibname}{References}
%\bibliography{literatur}
%
%
%
%%===================================================
%\chapter*{Supplement}
%\addcontentsline{toc}{chapter}{Supplementary}
%
%\newcommand{\beginsupplement}{
%        \setcounter{table}{0}
%        \renewcommand{\thetable}{S\arabic{table}}%
%        \setcounter{figure}{0}
%        \renewcommand{\thefigure}{S\arabic{figure}}%
%     }
% 
%\beginsupplement


\end{document}